\chapter{پیشنیاز های نصب و معرفی قسمت های مختلف}\label{ch:req}

\section{نرم‌افزار‌های کلی}
در این پروژه از جهت آنکه نسخه قبلی و پیشینی برای  آن نبوده است، به ناچار می‌بایست که کد آن از صفر تا صد آن به صورت دستی نوشته شود. از این‌رو، پیچیدگی های بسیار فراوان را به طور خاص در پی داشت. ابزار های زیادی نیز بنابه شرایط در آن استفاده شد که ارتباط بین آن ابزار ها و اجزا، بر این پیچیدگی پیاده سازی طرح افزوده بود.

ابزار های اصلی و کلی که در این پروژه استفاده شده بود، عبارتند از:

\begin{itemize}
	\item 
	\textbf{نرم افزار پری اسکن}
	\LTRfootnote{PreScan}
	، نسخه \lr{$8.5.0$}
	
	\item 
	\textbf{نرم افزار متلب}
	\LTRfootnote{Matlab}
	، نسخه \lr{R2017b}
	\item 
	\textbf{زبان برنامه نویسی پایتون}
	، نسخه \lr{$3.6.9$}
\end{itemize}

بنابراین برای راه اندازی مجدد کد این پروژه لازم است که موارد بالا روی کامپیوتر شخص به صورت کامل نصب باشد.

همچنین لازم به ذکر است که برخی ابزارات دیگر نیز در این پروژه استفاده شده است که احتمالا با نصب موارد بالا دیگر نیازی به نصب آن ها به صورت جداگانه نیست. هدف این ابزار ها ایجاد اتصال بین اجزای اصلی گفته شده است. این گروه شامل موارد زیر هستند:

\begin{itemize}
	\item 
	\textbf{سیمولینک}
	\LTRfootnote{Simulink}
	، جهت اتصال بین متلب و پری اسکن
	
	\item 
	\textbf{شبکه \lr{UDP}}
	\RTLfootnote{برای این منظور از ماژول \lr{socket} در پایتون استفاده شده است. }
	، جهت اتصال داده های پویا 
	\LTRfootnote{Dynamic Data}
	بین پایتون و سیمولینک
	
	\item 
	\textbf{موتور متلب}
	\LTRfootnote{\matlabengine}
	، جهت اتصال داده های ساکن
	\LTRfootnote{Static Data}
	بین پایتون و سیمولینک
\end{itemize}

در این فصل جزئیات بیشتری در مورد لزوم و دلیل استفاده از این ابزار ها بررسی می‌شود.




\section{پیشنیاز های پایتون}

کد پایتون در این پروژه شامل دو قسمت کلی زیر می‌شود.
\glspl{Action}

















