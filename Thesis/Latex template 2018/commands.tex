% در این فایل، دستورها و تنظیمات مورد نیاز، آورده شده است.
%-------------------------------------------------------------------------------------------------------------------
% در ورژن جدید زی‌پرشین برای تایپ متن‌های ریاضی، این سه بسته، حتماً باید فراخوانی شود.
\usepackage{amsthm,amssymb,amsmath,amsfonts}
% بسته‌ای برای تنطیم حاشیه‌های بالا، پایین، چپ و راست صفحه
\usepackage[top=30mm, bottom=30mm, left=25mm, right=30mm]{geometry}
% بسته‌‌ای برای ظاهر شدن شکل‌ها و تصاویر متن
\usepackage{graphicx}
%\usepackage{subfigure}
%\usepackage{subcaption}

\usepackage{color}
\usepackage[table]{xcolor}

%%%% Page color
\definecolor{cream}{rgb}{1.0, 0.99, 0.82}
\definecolor{floralwhite}{rgb}{1.0, 0.98, 0.94}
\definecolor{ghostwhite}{rgb}{0.97, 0.97, 1.0}
\definecolor{honeydew}{rgb}{0.94, 1.0, 0.94}
\definecolor{isabelline}{rgb}{0.96, 0.94, 0.93}
\definecolor{magnolia}{rgb}{0.97, 0.96, 1.0}
\definecolor{mintcream}{rgb}{0.96, 1.0, 0.98}
\definecolor{oldlace}{rgb}{0.99, 0.96, 0.9}
\definecolor{whitesmoke}{rgb}{0.96, 0.96, 0.96}
\definecolor{whitegray}{rgb}{0.98, 0.98, 0.98}
\definecolor{LightCyan}{rgb}{0.88,1,1}

\usepackage{listings}

\definecolor{dkgreen}{rgb}{0,0.6,0}
\definecolor{gray}{rgb}{0.5,0.5,0.5}
\definecolor{lightgray}{rgb}{0.8,0.8,0.8}
\definecolor{mauve}{rgb}{0.58,0,0.82}

\lstset{frame=tBLr,
%	frameround=fttt,
	language=python,
	aboveskip=3mm,
	belowskip=3mm,
	showstringspaces=false,
	columns=flexible,
	basicstyle={\small\ttfamily},
	lineskip=0.5em,
	xleftmargin=23pt,
	xrightmargin=9pt,
	framexleftmargin=17pt,
	framexrightmargin=5pt,
	framexbottommargin=2pt,
	numbers=left,
	firstnumber=1,
	numberstyle=\tiny\color{gray},
	keywordstyle=\color{blue},
	commentstyle=\color{dkgreen},
	stringstyle=\color{mauve},
	rulecolor=\color{black},
	breaklines=true,
	breakatwhitespace=true,
	tabsize=4
}
%\lstset{
%	language=python,
%	frame=trBL,  
%%	frameround=fttt,
%	numberstyle=\tiny\color{gray},
%	keywordstyle=\color{blue},
%	commentstyle=\color{dkgreen},
%	stringstyle=\color{mauve},
%	numbers=left,
%	numberstyle=\scriptsize,
%	showspaces=false,         
%	showtabs=false,       
%	xleftmargin=23pt,
%	xrightmargin=9pt,
%	framexleftmargin=17pt,
%	framexrightmargin=5pt,
%	framexbottommargin=2pt,
%	showstringspaces=false ,
%	breaklines=true,
%	texcl=true,
%	basicstyle=\setLTR\footnotesize\ttfamily,
%	lineskip=0.5em,
%	commentstyle= \itshape,
%	belowcaptionskip=12pt,
%	aboveskip=15pt,
%	belowskip=15pt
%}  
\newcommand{\code}[2][python]{
	\begin{latin}
%		\noindent #2
		\lstinputlisting[language=#1]{code/#2}
	\end{latin}
}
\definecolor{backgroundcode}{HTML}{FBFBFB}

\colorlet{punct}{red!60!black}
%\definecolor{background}{HTML}{EEEEEE}
\definecolor{delim}{RGB}{20,105,176}
\colorlet{numb}{magenta!60!black}
\definecolor{darkbrown}{rgb}{0.4, 0.26, 0.13}
\definecolor{darkblue}{rgb}{0.0, 0.0, 0.55}




\lstdefinelanguage{json}{
%	basicstyle=\normalfont\ttfamily,
	basicstyle=\scriptsize\ttfamily,
%	numbers=left,	
	numbers=none,
	numberstyle=\scriptsize,
	stepnumber=1,
	numbersep=8pt,
	showstringspaces=false,
	breaklines=true,
	breakatwhitespace=false,
	breakindent=0pt,
	frame=single,
	xleftmargin=10pt,
	xrightmargin=10pt,
	framexleftmargin=5pt,
	framexrightmargin=5pt,
		string=[s]{"}{"},
		comment=[l]{:\ "},
		morecomment=[l]{:"},
	moredelim=**[is][\color{red}]{@}{@},
	backgroundcolor=\color{backgroundcode},
	stringstyle=\color{black}, % style of strings
	literate=%
	*{0}{{{\color{numb}0}}}{1}
	{1}{{{\color{numb}1}}}{1}
	{2}{{{\color{numb}2}}}{1}
	{3}{{{\color{numb}3}}}{1}
	{4}{{{\color{numb}4}}}{1}
	{5}{{{\color{numb}5}}}{1}
	{6}{{{\color{numb}6}}}{1}
	{7}{{{\color{numb}7}}}{1}
	{8}{{{\color{numb}8}}}{1}
	{9}{{{\color{numb}9}}}{1}
	{e+}{{{\color{darkbrown}{e+}}}}{1}	
	{e-}{{{\color{darkbrown}{e-}}}}{1}
	{:}{{{\color{punct}{:}}}}{1}
	{,}{{{\color{punct}{,}}}}{1}
	{.}{{{\color{darkbrown}{.}}}}{1}
	{"}{{{\color{darkblue}{"}}}}{1}
	{'}{{{\color{red}{'}}}}{1}
   {\{}{{{\color{delim}\{}}}{1}
   {\}}{{{\color{delim}{\}}}}}{1}
	{[}{{{\color{delim}{[}}}}{1}
	{]}{{{\color{delim}{]}}}}{1},
}

%
%\definecolor{eclipseStrings}{RGB}{42,0.0,255}
%\definecolor{eclipseKeywords}{RGB}{127,0,85}
%\colorlet{numb}{magenta!60!black}
%
%\lstdefinelanguage{json}{
%	basicstyle=\normalfont\ttfamily,
%	commentstyle=\color{dkgreen}, % style of comment
%	stringstyle=\color{mauve}, % style of strings
%%	numbers=left,
%	numbers=none,
%	numberstyle=\scriptsize,
%	stepnumber=1,
%	numbersep=8pt,
%	showstringspaces=false,
%	breaklines=true,
%	frame=single,
%	backgroundcolor=\color{backgroundcode}, %only if you like
%	string=[s]{"}{"},
%	comment=[l]{:\ "},
%	morecomment=[l]{:"},
%	literate=
%	*{0}{{{\color{numb}0}}}{1}
%	{1}{{{\color{numb}1}}}{1}
%	{2}{{{\color{numb}2}}}{1}
%	{3}{{{\color{numb}3}}}{1}
%	{4}{{{\color{numb}4}}}{1}
%	{5}{{{\color{numb}5}}}{1}
%	{6}{{{\color{numb}6}}}{1}
%	{7}{{{\color{numb}7}}}{1}
%	{8}{{{\color{numb}8}}}{1}
%	{9}{{{\color{numb}9}}}{1}
%}
%




%بسته‌ای برای تنظیم فاصله عمودی خط‌های متن
\usepackage{setspace}
\usepackage{titletoc}
\usepackage{subfigure}
\usepackage[subfigure]{tocloft}
%\usepackage{tocloft}
%\setlength{\abovecaptionskip}{10pt plus 3pt minus 2pt} % Chosen fairly arbitrarily


\usepackage{array}%,makecell, cellspace}
%\newcommand{\tablewidth}{\def\arraystretch{5}}
\newcommand{\tableset}[2][2.0]{\centering\renewcommand{\arraystretch}{#1}{#2}}
\newcolumntype{C}[1]{>{\centering\arraybackslash}p{#1}}



\usepackage{tikz}
\newcommand*\circled[1]{\tikz[baseline=(char.base)]{\node[shape=circle,fill=blue!20,draw,inner sep=2pt] (char) {#1};}}

%با فعال کردن بسته زیر فوت‌نوت‌ها در هر صفحه ریست می‌شوند. حالت پیش‌فرض آن ریست شدن در هر فصل می‌باشد.
%\usepackage[perpage]{footmisc}
\usepackage[inline]{enumitem}
\newlist{alphinline}{enumerate*}{1}
\setlist[alphinline]{label=(\alph*)}
\newlist{enuminline}{enumerate*}{1}
\setlist[enuminline]{label=(\arabic*)}
\newlist{circlelist}{enumerate}{1}
\setlist[circlelist,1]{label=\protect\circled{\arabic*}}
\newlist{alphabetlist}{enumerate}{1}
\setlist[alphabetlist]{label=\alph*)}


%\usepackage{titlesec}
% بسته‌ و دستوراتی برای ایجاد لینک‌های رنگی با امکان جهش
\usepackage[pagebackref=false,colorlinks,linkcolor=blue,citecolor=red]{hyperref}
\usepackage[nameinlink]{cleveref}%capitalize,,noabbrev
 \AtBeginDocument{%
    \crefname{equation}{برابری}{equations}%
    \crefname{chapter}{فصل}{chapters}%
    \crefname{section}{بخش}{sections}%
    \crefname{appendix}{پیوست}{appendices}%
    \crefname{enumi}{مورد}{items}%
    \crefname{footnote}{زیرنویس}{footnotes}%
    \crefname{figure}{شکل}{figures}%
    \crefname{table}{جدول}{tables}%
    \crefname{theorem}{قضیه}{theorems}%
    \crefname{lemma}{لم}{lemmas}%
    \crefname{corollary}{نتیجه}{corollaries}%
    \crefname{proposition}{گزاره}{propositions}%
    \crefname{definition}{تعریف}{definitions}%
    \crefname{result}{نتیجه}{results}%
    \crefname{example}{مثال}{examples}%
    \crefname{remark}{نکته}{remarks}%
    \crefname{note}{یادداشت}{notes}%
}
% چنانچه قصد پرینت گرفتن نوشته خود را دارید، خط بالا را غیرفعال و  از دستور زیر استفاده کنید چون در صورت استفاده از دستور زیر‌‌، 
% لینک‌ها به رنگ سیاه ظاهر خواهند شد که برای پرینت گرفتن، مناسب‌تر است
%\usepackage[pagebackref=false]{hyperref}
% بسته‌ لازم برای تنظیم سربرگ‌ها
\usepackage{fancyhdr}
% بسته‌ای برای ظاهر شدن «مراجع»  در فهرست مطالب
\usepackage[nottoc]{tocbibind}
% دستورات مربوط به ایجاد نمایه
\usepackage{makeidx,multicol}
\setlength{\columnsep}{1.5cm}

%%%%%%%%%%%%%%%%%%%%%%%%%%
\usepackage{verbatim}
\makeindex
\usepackage{sectsty}

\usepackage[xindy,acronym,nonumberlist=true]{glossaries}



% فراخوانی بسته زی‌پرشین و تعریف قلم فارسی و انگلیسی
\usepackage{xepersian}%[extrafootnotefeatures]
\SepMark{-}
%حتماً از تک لایو 2014 استفاده کنید.
\settextfont[Scale=1.2]{B Nazanin}
\setlatintextfont{Times New Roman}
\renewcommand{\labelitemi}{$\bullet$}
%%%%%%%%%%%%%%%%%%%%%%%%%%
% چنانچه می‌خواهید اعداد در فرمول‌ها، انگلیسی باشد، خط زیر را غیرفعال کنید.
%در غیر اینصورت حتماً فونت PGaramond را نصب کنید.
\setdigitfont[Scale=1.1]{Yas}
%%%%%%%%%%%%%%%%%%%%%%%%%%
% تعریف قلم‌های فارسی اضافی برای استفاده در بعضی از قسمت‌های متن
\defpersianfont\nastaliq[Scale=2]{IranNastaliq}
\defpersianfont\chapternumber[Scale=3]{B Nazanin}
%\chapterfont{\centering}%
%%%%%%%%%%%%%%%%%%%%%%%%%%
% دستوری برای تغییر نام کلمه «اثبات» به «برهان»
\renewcommand\proofname{\textbf{برهان}}

% دستوری برای تغییر نام کلمه «کتاب‌نامه» به «منابع و مراجع«
\renewcommand{\bibname}{منابع و مراجع}


% Headings for every page of ToC, LoF and Lot
\setlength{\cftbeforetoctitleskip}{-1.2em}
\setlength{\cftbeforelottitleskip}{-1.2em}
\setlength{\cftbeforeloftitleskip}{-1.2em}
\setlength{\cftaftertoctitleskip}{-1em}
\setlength{\cftafterlottitleskip}{-1em}
\setlength{\cftafterloftitleskip}{-1em}
%%\makeatletter
%%%%\renewcommand{\l@chapter}{\@dottedtocline{1}{1em\bfseries}{1em}}
%%%%\renewcommand{\l@section}{\@dottedtocline{2}{2em}{2em}}
%%%%\renewcommand{\l@subsection}{\@dottedtocline{3}{3em}{3em}}
%%%%\renewcommand{\l@subsubsection}{\@dottedtocline{4}{4em}{4em}}
%%%%\makeatother


\newcommand\tocheading{\par عنوان\hfill صفحه \par}
\newcommand\lofheading{\hspace*{.5cm}\figurename\hfill صفحه \par}
\newcommand\lotheading{\hspace*{.5cm}\tablename\hfill صفحه \par}

\renewcommand{\cftchapleader}{\cftdotfill{\cftdotsep}}
\renewcommand{\cfttoctitlefont}{\hspace*{\fill}\LARGE\bfseries}%\Large
\renewcommand{\cftaftertoctitle}{\hspace*{\fill}}
\renewcommand{\cftlottitlefont}{\hspace*{\fill}\LARGE\bfseries}%\Large
\renewcommand{\cftafterlottitle}{\hspace*{\fill}}
\renewcommand{\cftloftitlefont}{\hspace*{\fill}\LARGE\bfseries}
\renewcommand{\cftafterloftitle}{\hspace*{\fill}}

%%%%%%%%%%%%%%%%%%%%%%%%%%
% تعریف و نحوه ظاهر شدن عنوان قضیه‌ها، تعریف‌ها، مثال‌ها و ...
%برای شماره گذاری سه تایی قضیه ها
\theoremstyle{definition}
\newtheorem{definition}{تعریف}[section]
\newtheorem{remark}[definition]{نکته}
\newtheorem{note}[definition]{یادداشت}
\newtheorem{example}[definition]{نمونه}
\newtheorem{question}[definition]{سوال}
\newtheorem{remember}[definition]{یاداوری}
\theoremstyle{theorem}
\newtheorem{theorem}[definition]{قضیه}
\newtheorem{lemma}[definition]{لم}
\newtheorem{proposition}[definition]{گزاره}
\newtheorem{corollary}[definition]{نتیجه}
%%%%%%%%%%%%%%%%%%%%%%%%
%%%%%%%%%%%%%%%%%%%
%%% برای شماره گذاری چهارتایی قضیه ها و ...
%%\newtheorem{definition1}[subsubsection]{تعریف}
%%\newtheorem{theorem1}[subsubsection]{قضیه}
%%\newtheorem{lemma1}[subsubsection]{لم}
%%\newtheorem{proposition1}[subsubsection]{گزاره}
%%\newtheorem{corollary1}[subsubsection]{نتیجه}
%%\newtheorem{remark1}[subsubsection]{نکته}
%%\newtheorem{example1}[subsubsection]{مثال}
%%\newtheorem{question1}[subsubsection]{سوال}

%%%%%%%%%%%%%%%%%%%%%%%%%%%%

% دستورهایی برای سفارشی کردن صفحات اول فصل‌ها
\makeatletter
\newcommand\mycustomraggedright{%
 \if@RTL\raggedleft%
 \else\raggedright%
 \fi}
\def\@makechapterhead#1{%
\thispagestyle{style1}
\vspace*{20\p@}%
{\parindent \z@ \mycustomraggedright
\ifnum \c@secnumdepth >\m@ne
\if@mainmatter

\bfseries{\Huge \@chapapp}\small\space {\chapternumber\thechapter}
\par\nobreak
\vskip 0\p@
\fi
\fi
\interlinepenalty\@M 
\Huge \bfseries #1\par\nobreak
\vskip 120\p@

}

%\thispagestyle{empty}
\newpage}
\bidi@patchcmd{\@makechapterhead}{\thechapter}{\tartibi{chapter}}{}{}
\bidi@patchcmd{\chaptermark}{\thechapter}{\tartibi{chapter}}{}{}
\makeatother

\pagestyle{fancy}
\renewcommand{\chaptermark}[1]{\markboth{\chaptername~\tartibi{chapter}: #1}{}}

\fancypagestyle{style1}{
\fancyhf{} 
\fancyfoot[c]{\thepage}
\fancyhead[R]{\leftmark}%
\renewcommand{\headrulewidth}{1.2pt}
}


\fancypagestyle{style2}{
\fancyhf{}
\fancyhead[R]{چکیده}
\fancyfoot[C]{\thepage{}}
\renewcommand{\headrulewidth}{1.2pt}
}

\fancypagestyle{style3}{%
  \fancyhf{}%
  \fancyhead[R]{فهرست نمادها}
  \fancyfoot[C]{\thepage}%
  \renewcommand{\headrulewidth}{1.2pt}%
}

\fancypagestyle{style4}{%
  \fancyhf{}%
  \fancyhead[R]{فهرست جداول}
  \fancyfoot[C]{\thepage}%
  \renewcommand{\headrulewidth}{1.2pt}%
}

\fancypagestyle{style5}{%
  \fancyhf{}%
  \fancyhead[R]{فهرست اشکال}
  \fancyfoot[C]{\thepage}%
  \renewcommand{\headrulewidth}{1.2pt}%
}

\fancypagestyle{style6}{%
  \fancyhf{}%
  \fancyhead[R]{فهرست مطالب}
  \fancyfoot[C]{\thepage}%
  \renewcommand{\headrulewidth}{1.2pt}%
}

\fancypagestyle{style7}{%
  \fancyhf{}%
  \fancyhead[R]{نمایه}
  \fancyfoot[C]{\thepage}%
  \renewcommand{\headrulewidth}{1.2pt}%
}

\fancypagestyle{style8}{%
  \fancyhf{}%
  \fancyhead[R]{منابع و مراجع}
  \fancyfoot[C]{\thepage}%
  \renewcommand{\headrulewidth}{1.2pt}%
}
\fancypagestyle{style9}{%
  \fancyhf{}%
  \fancyhead[R]{واژه‌نامه‌ی فارسی به انگلیسی}
  \fancyfoot[C]{\thepage}%
  \renewcommand{\headrulewidth}{1.2pt}%
}
%


%دستور حذف نام لیست تصاویر و لیست جداول از فهرست مطالب
\newcommand*{\BeginNoToc}{%
  \addtocontents{toc}{%
    \edef\protect\SavedTocDepth{\protect\the\protect\value{tocdepth}}%
  }%
  \addtocontents{toc}{%
    \protect\setcounter{tocdepth}{-10}%
  }%
}
\newcommand*{\EndNoToc}{%
  \addtocontents{toc}{%
    \protect\setcounter{tocdepth}{\protect\SavedTocDepth}%
  }%
}
\newcounter{savepage}
\renewcommand{\listfigurename}{فهرست اشکال}
\renewcommand{\listtablename}{فهرست جداول}
%\renewcommand\cftsecleader{\cftdotfill{\cftdotsep}}
%%%%%%%%%%%%%%%%%%%%%%%%%%%%%
%%%%%%%%%%%%%%%%%%%%%%%%%%%%




%%%%%% ============================================================================================================
%\usepackage[xindy,acronym,nonumberlist=true]{glossaries}
%
%\usepackage{xepersian}%[extrafootnotefeatures]
%%%%%% ============================================================================================================

%%% تنظیمات مربوط به بسته  glossaries
%%% تعریف استایل برای واژه نامه فارسی به انگلیسی، در این استایل واژه‌های فارسی در سمت راست و واژه‌های انگلیسی در سمت چپ خواهند آمد. از حالت گروه ‌بندی استفاده می‌کنیم، 
%%% یعنی واژه‌ها در گروه‌هایی به ترتیب حروف الفبا مرتب می‌شوند، مثلا:
%%% الف
%%% افتصاد ................................... Economy
%%% اشکال ........................................ Failure
%%% ش
%%% شبکه ...................................... Network
\newglossarystyle{myFaToEn}{%
	\renewenvironment{theglossary}{}{}
	\renewcommand*{\glsgroupskip}{\vskip 10mm}
	\renewcommand*{\glsgroupheading}[1]{\subsection*{\glsgetgrouptitle{##1}}}
	\renewcommand*{\glossentry}[2]{\noindent\glsentryname{##1}\dotfill\space \glsentrytext{##1}
		
	}
}

%% % تعریف استایل برای واژه نامه انگلیسی به فارسی، در این استایل واژه‌های فارسی در سمت راست و واژه‌های انگلیسی در سمت چپ خواهند آمد. از حالت گروه ‌بندی استفاده می‌کنیم، 
%% % یعنی واژه‌ها در گروه‌هایی به ترتیب حروف الفبا مرتب می‌شوند، مثلا:
%% % E
%%% Economy ............................... اقتصاد
%% % F
%% % Failure................................... اشکال
%% %N
%% % Network ................................. شبکه

\newglossarystyle{myEntoFa}{%
	%%% این دستور در حقیقت عملیات گروه‌بندی را انجام می‌دهد. بدین صورت که واژه‌ها در بخش‌های جداگانه گروه‌بندی می‌شوند، 
	%%% عنوان بخش همان نام حرفی است که هر واژه در آن گروه با آن شروع شده است. 
	\renewenvironment{theglossary}{}{}
	\renewcommand*{\glsgroupskip}{\vskip 10mm}
	\renewcommand*{\glsgroupheading}[1]{\begin{latin} \subsection*{\glsgetgrouptitle{##1}} \end{latin}}
	%%% در این دستور نحوه نمایش واژه‌ها می‌آید. در این جا واژه فارسی در سمت راست و واژه انگلیسی در سمت چپ قرار داده شده است، و بین آن با نقطه پر می‌شود. 
	\renewcommand*{\glossentry}[2]{\noindent\glsentrytext{##1}\dotfill\space \glsentryname{##1}
		
	}
}

%%% تعیین استایل برای فهرست اختصارات
\newglossarystyle{myAbbrlist}{%
	%%% این دستور در حقیقت عملیات گروه‌بندی را انجام می‌دهد. بدین صورت که اختصارات‌ در بخش‌های جداگانه گروه‌بندی می‌شوند، 
	%%% عنوان بخش همان نام حرفی است که هر اختصار در آن گروه با آن شروع شده است. 
	\renewenvironment{theglossary}{}{}
	\renewcommand*{\glsgroupskip}{\vskip 10mm}
	\renewcommand*{\glsgroupheading}[1]{\begin{latin} \subsection*{\glsgetgrouptitle{##1}} \end{latin}}
	%%% در این دستور نحوه نمایش اختصارات می‌آید. در این جا حالت کوچک اختصار در سمت چپ و حالت بزرگ در سمت راست قرار داده شده است، و بین آن با نقطه پر می‌شود. 
	\renewcommand*{\glossentry}[2]{\noindent\glsentrytext{##1}\dotfill\space \Glsentrylong{##1}
		
	}
	%%% تغییر نام محیط abbreviation به فهرست اختصارات
	\renewcommand*{\acronymname}{\rl{فهرست اختصارات}}
}

%%% برای اجرا xindy بر روی فایل .tex و تولید واژه‌نامه‌ها و فهرست اختصارات و فهرست نمادها یکسری  فایل تعریف شده است.‌ Latex داده های مربوط به واژه نامه و .. را در این 
%%%  فایل‌ها نگهداری می‌کند. مهم‌ترین option‌ این قسمت این است که 
%%% عنوان واژه‌نامه‌ها و یا فهرست اختصارات و یا فهرست نمادها را می‌توانید در این‌جا مشخص کنید. 
%%% در این جا عباراتی مثل glg، gls، glo و ... پسوند فایل‌هایی است که برای xindy بکار می‌روند. 
\newglossary[glg]{english}{gls}{glo}{واژه‌نامه انگلیسی به فارسی}
\newglossary[blg]{persian}{bls}{blo}{واژه‌نامه فارسی به انگلیسی}
\makeglossaries
\glsdisablehyper
%%% تعاریف مربوط به تولید واژه نامه و فهرست اختصارات و فهرست نمادها
%%%  در این فایل یکسری دستورات عمومی برای وارد کردن واژه‌نامه آمده است.
%%%  به دلیل این‌که قرار است این دستورات پایه‌ای را بازنویسی کنیم در این‌جا تعریف می‌کنیم. 
\let\oldgls\gls
\let\oldglspl\glspl

\makeatletter

\renewrobustcmd*{\gls}{\@ifstar\@msgls\@mgls}
\newcommand*{\@mgls}[1] {\ifthenelse{\equal{\glsentrytype{#1}}{english}}{\oldgls{#1}\glsuseri{f-#1}}{\lr{\oldgls{#1}}}}
\newcommand*{\@msgls}[1]{\ifthenelse{\equal{\glsentrytype{#1}}{english}}{\glstext{#1}\glsuseri{f-#1}}{\lr{\glsentryname{#1}}}}

\renewrobustcmd*{\glspl}{\@ifstar\@msglspl\@mglspl}
\newcommand*{\@mglspl}[1] {\ifthenelse{\equal{\glsentrytype{#1}}{english}}{\oldglspl{#1}\glsuseri{f-#1}}{\oldglspl{#1}}}
\newcommand*{\@msglspl}[1]{\ifthenelse{\equal{\glsentrytype{#1}}{english}}{\glsplural{#1}\glsuseri{f-#1}}{\glsentryplural{#1}}}

\makeatother

\newcommand{\newword}[4]{
	\newglossaryentry{#1}     {type={english},name={\lr{#2}},plural={#4},text={#3},description={}}
	\newglossaryentry{f-#1} {type={persian},name={#3},text={\lr{#2}},description={}}
}

%%% بر طبق این دستور، در اولین باری که واژه مورد نظر از واژه‌نامه وارد شود، پاورقی زده می‌شود. 
\defglsentryfmt[english]{\glsgenentryfmt\ifglsused{\glslabel}{}{\LTRfootnote{\glsentryname{\glslabel}}}}

%%% بر طبق این دستور، در اولین باری که واژه مورد نظر از فهرست اختصارات وارد شود، پاورقی زده می‌شود. 
\defglsentryfmt[acronym]{\glsentryname{\glslabel}\ifglsused{\glslabel}{}{\LTRfootnote{\glsentrydesc{\glslabel}}}}


%%%%%% ============================================================================================================

%%============================ دستور برای قرار دادن فهرست اختصارات 
\newcommand{\printabbreviation}{
	\cleardoublepage
	\phantomsection
	\baselineskip=.75cm
	%% با این دستور عنوان فهرست اختصارات به فهرست مطالب اضافه می‌شود. 
	\addcontentsline{toc}{chapter}{فهرست اختصارات}
	\setglossarystyle{myAbbrlist}
	\begin{latin}
		\Oldprintglossary[type=acronym]	
	\end{latin}
	\clearpage
}%

\newcommand{\printacronyms}{\printabbreviation}
%%% در این جا محیط هر دو واژه نامه را باز تعریف کرده ایم، تا اولا مشکل قرار دادن صفحه اضافی را حل کنیم، ثانیا عنوان واژه نامه ها را با دستور addcontentlist وارد فهرست مطالب کرده ایم.
\let\Oldprintglossary\printglossary
\renewcommand{\printglossary}{
	\let\appendix\relax
	%% تنظیم کننده فاصله بین خطوط در این قسمت
	\clearpage
	\phantomsection
	%% این دستور موجب این می‌شود که واژه‌نامه‌ها در  حالت دو ستونی نوشته شود. 
	\twocolumn{}
	%% با این دستور عنوان واژه‌نامه به فهرست مطالب اضافه می‌شود. 
	\addcontentsline{toc}{chapter}{واژه نامه انگلیسی به فارسی}
	\setglossarystyle{myEntoFa}
	\Oldprintglossary[type=english]
	
	\clearpage
	\phantomsection
	%% با این دستور عنوان واژه‌نامه به فهرست مطالب اضافه می‌شود. 
	\addcontentsline{toc}{chapter}{واژه نامه فارسی به انگلیسی}
	\setglossarystyle{myFaToEn}
	\Oldprintglossary[type=persian]
	\onecolumn{}
}%

% ===============================================================================================================


%xindy -L persian-variant1 -C utf8 -I xindy -M %.xdy -t %.glg -o %.gls %.glo | 
%xindy -L persian-variant1 -C utf8 -I xindy -M %.xdy -t %.blg -o %.bls %.blo |
%xindy -L english -C utf8 -I xindy -M %.xdy -t %.alg -o %.acr %.acn


%xindy -M texindy -C utf8 -L persian-varient2 %.idx

%!main=AUTthesis
%===============
%%% نحوه تعریف واژگان 

\newword{RandomVariable}{Random Variable}
{متغیر تصادفی}{متغیرهای تصادفی}

\newword{w:action}{Action}{حرکت}{حرکت}
\newword{w:state}{State}{حرکت}{حرکت}
\newword{w:reward}{Reward}{امتیاز}{امتیاز}
\newword{w:env}{Environment}{محیط}{محیط}
\newword{w:agent}{Agent}{عامل}{عامل}

\newword{w:actionspace}{Action Space}{فضای حرکت}{فضای حرکت}
\newword{w:obsspace}{Observation Space}{فضای مشاهده}{فضای مشاهده}

\newword{w:matlab}{Matlab}{نرم‌افزار متلب}{متلب}
\newword{w:simulink}{Simulink}{سیمولینک}{سیمولینک}
\newword{w:matlabengine}{Matlab Engine}{موتور متلب}{موتور متلب}
\newword{w:anaconda}{Anaconda}{نرم‌افزار آناکوندا}{آناکوندا}
\newword{w:prescan}{PreScan}{نرم‌افزار پری‌اسکن}{پری‌اسکن}
\newword{w:timeout}{Timeout}{سقف زمانی}{سقف زمانی}
\newword{w:episode}{Episode}{اپیزود}{اپیزود}
\newword{w:dyndata}{Dynamic Data}{داده‌های پویا}{داده‌های پویا}
\newword{w:drl}{Deep Reinforcement Learning}{یادگیری تقویتی عمیق}{یادگیری تقویتی عمیق}
\newword{w:ml}{Machine Learning}{یادگیری ماشین}{یادگیری ماشین}
%\newword{w:}{}{}{}
%\newword{w:}{}{}{}
%\newword{w:}{}{}{}
%\newword{w:}{}{}{}
%\newword{w:}{}{}{}
%\newword{w:}{}{}{}
%\newword{w:}{}{}{}
%\newword{w:}{}{}{}
%\newword{w:}{}{}{}
%\newword{w:}{}{}{}
%\newword{w:}{}{}{}
%\newword{w:}{}{}{}
%\newword{w:}{}{}{}




%%%%%% ============================================================================================================

%%% نحوه تعریف اختصارات
%\newacronym{DFT}{DFT}{Discrete Fourier Transform}
%\newacronym{CDMA}{CDMA}{Code Division Multiplexing Access}
%\newacronym{BAN}{BAN}{Body Area Network}



%%%%%% ============================================================================================================


\newcommand{\w}[1]{\glspl{w:#1}}
\newcommand{\ws}[1]{\glspl*{w:#1}}

%%%%%% ============================================================================================================
