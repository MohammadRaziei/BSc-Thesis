%% -!TEX root = AUTthesis.tex
% در این فایل، عنوان پایان‌نامه، مشخصات خود، متن تقدیمی‌، ستایش، سپاس‌گزاری و چکیده پایان‌نامه را به فارسی، وارد کنید.
% توجه داشته باشید که جدول حاوی مشخصات پروژه/پایان‌نامه/رساله و همچنین، مشخصات داخل آن، به طور خودکار، درج می‌شود.
%%%%%%%%%%%%%%%%%%%%%%%%%%%%%%%%%%%%
% دانشکده، آموزشکده و یا پژوهشکده  خود را وارد کنید
\faculty{دانشکده مهندسی برق}
% گرایش و گروه آموزشی خود را وارد کنید
\department{گرایش مخابرات}
% عنوان پایان‌نامه را وارد کنید
\fatitle{ماشین های خودران- 
	\\[0.75cm]
	با استفاده از یادگیری تقویتی}
% نام استاد(ان) راهنما را وارد کنید
\firstsupervisor{دکتر وحید پوراحمدی}
\secondsupervisor{دکتر حمیدرضا امین‌داور}
% نام استاد(دان) مشاور را وارد کنید. چنانچه استاد مشاور ندارید، دستور پایین را غیرفعال کنید.
%\firstadvisor{دکتر حمیدرضا امین‌داور}
%\secondadvisor{استاد مشاور دوم}
% نام نویسنده را وارد کنید
\name{محمد }
% نام خانوادگی نویسنده را وارد کنید
\surname{رضیئی فیجانی}
%%%%%%%%%%%%%%%%%%%%%%%%%%%%%%%%%%
\thesisdate{شهریور ۱۳۹۸}

% چکیده پایان‌نامه را وارد کنید
\fa-abstract{
%در اين قسمت چكيده پایان نامه نوشته مي‌شو‌د‌.‌ چكيده بايد جامع و بيان‌كننده‌ خلاصه‌اي از اقدامات انجام‌شده باشد. در چكيده باید از ارجاع به مرجع و ذكر روابط رياضي، بيان تاريخچه و تعريف مسئله خودداري ‌شود. 
%\\
خودران کردن خودرو و ساخن خودرو های هوشمند یکی از اهداف بزرگی است که محققان فراوانی در دنیا بر روی این موضوع کار می‌کنند و شرکت های بزرگی نیز از آن ها حمایت می‌کنند. 
\\
امروزه الگوریتم های یادگیری ماشین در بسیاری از علوم مهندسی و غیر مهندسی مانند کامپیوتر، مخابرات، کنترل، اقتصاد، پردازش تصویر، پردازش صوت و سیگنال، مهندسی پزشکی، علوم اعصاب، خودرو سازی و ... کاربرد فراوان دارد. از این رو محققان این حوزه در تلاش هستند تا آن روش ها را بهبود بخشند و یا آن‌ها را در علوم یاد شده، پیاده سازند. 
\\
یکی از شاخه های یادگیری ماشین، یادگیری تقویتی است. 
در این شاخه سعی شده است تا یادگیرنده، همان گونه‌ای یاد بگیرد که انسان یاد می‌گیرد. این روش با الگوهای با ناظر و یا بدون ناظر متفاوت است و چیزی به عنوان ناظر وجود ندارد و صرفا امتیاز ها هستند که تعیین کننده و راهنما هستند. 
\\
ترکیب این دو حوزه یعنی پیاده سازی خودرو های خودران با استفاده از الگوریتم های یادگیری تقویتی، بسیار جالب خواهد بود.  فرضیه ای در این حوزه یادگیری تقویتی به نام فرضیه امیتاز ها وجود دارد که بیان می کند هر رخدادی را می توان با بیشینه کردن امید امتیاز تجمعی، توصیف کرد. بنابر این ادعا در این پروژه سعی شد با تعریف مناسب امتیاز ها و سایر پارامتر ها به 
خودرو آموخت تا خودران شود. 
\\
به صورت کلی این پروژه دارای دولایه می‌باشد؛ لایه الگوریتم و لایه شبیه ساز. لایه الگوریتم با استفاده از پایتون و لایه شبیه ساز با استفاده از نرم افزار پری‌اسکن در محیط سیمولینک انجام شده است. نوشتن محیط شبیه ساز از جمله دستاورد های بزرگ این پروژه می‌باشد.
%
}




% کلمات کلیدی پایان‌نامه را وارد کنید
\keywords{کلیدواژه اول، ...، کلیدواژه پنجم (نوشتن سه تا پنج واژه کلیدی ضروری است)}



\AUTtitle
%%%%%%%%%%%%%%%%%%%%%%%%%%%%%%%%%%
\vspace*{7cm}
\thispagestyle{empty}
\begin{center}
\includegraphics[height=5cm,width=12cm]{besm}
\end{center}