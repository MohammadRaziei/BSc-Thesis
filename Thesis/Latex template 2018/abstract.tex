در این پروژه سعی شد تا با استفاده از الگوریتم های یادگیری تقویتی، خودرو خودران تولید شود. در این جا اتومبیل در نقش «عامل» در ادبیات یادگیری تقویتی قرار دارد. در این پروژه با تعریف مناسب «امتیاز»ها ، «مشاهده»ها و همچنین پارامترهای الگوریتم های مختلف سعی شد تا هرچه بهتر و بیشتر به این هدف نزدیک شود. گفتنی است که تمامی مراحل کار از قسمت الگوریتم و تعریف پارامتر های ذکر شده تا نوشتن خود نرم‌افزار محیط شبیه سازی از جمله دستاوردهای این پروژه محسوب می‌شود.

در فصل \ref{ch:rl} توضیح بسیار مختصری در مورد خود مفاهیم یادگیری تقویتی می‌شود. فصل \ref{ch:resault} راه اندازی کد و نتایج حاصل این پروژه را نشان می‌دهد. پیش از راه اندازی باید باتوجه به فصل \ref{ch:req}، پیشنیاز های نصب آن تهیه و نصب گردند. همچنین آن فصل توضیح مختصری در مورد چیستی هریک از آن پیشنیاز ها ارایه کرده است. در فصل \ref{ch:alg}، نحوه تعریف پارامترهای الگوریتم یادگیری تقویتی به صورت کامل بسط داده شده اند. فصل \ref{ch:fani}، جزییات پیاده سازی محیط شبیه سازی را نشان می‌دهد و بر روی جزییات فنی آن تمرکز دارد.



